\documentclass[12pt]{article}
\usepackage{amsmath}
\usepackage{graphicx}
\usepackage{hyperref}
\usepackage[latin1]{inputenc}
\RequirePackage{tabularx}

\begin{document}
\section*{package-json-to-template 1.0.3 by Livio Brunner }

Parses package.json data and prints it in the given template (HTML, LaTeX, etc.)

\subsection*{Dependencies}

\begin{tabularx}{\textwidth}{p{0.25\textwidth} p{0.75\textwidth}}
    {\it Name} & {\it Description}\\
    \hline
    
    api-npm & This module will do query NPM registry to fetch any NPM module details including statistics.\\
    
    commander & the complete solution for node.js command-line programs\\
    
    is-object & Checks whether a value is an object\\
    
    nunjucks & A powerful templating engine with inheritance, asynchronous control, and more (jinja2 inspired)\\
    
    promise-pipeline & nodejs pipeline\\
     
\end{tabularx}

\subsection*{DevDependencies}

\begin{tabularx}{\textwidth}{p{0.25\textwidth} p{0.75\textwidth}}
    {\it Name} & {\it Description}\\
    \hline
    
    mocha & simple, flexible, fun test framework\\
     
\end{tabularx}


\newpage

\end{document}